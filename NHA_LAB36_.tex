% Options for packages loaded elsewhere
\PassOptionsToPackage{unicode}{hyperref}
\PassOptionsToPackage{hyphens}{url}
%
\documentclass[
]{article}
\title{NHA\_LAB36}
\author{Nancy Hernández Ambris}
\date{8/3/2022}

\usepackage{amsmath,amssymb}
\usepackage{lmodern}
\usepackage{iftex}
\ifPDFTeX
  \usepackage[T1]{fontenc}
  \usepackage[utf8]{inputenc}
  \usepackage{textcomp} % provide euro and other symbols
\else % if luatex or xetex
  \usepackage{unicode-math}
  \defaultfontfeatures{Scale=MatchLowercase}
  \defaultfontfeatures[\rmfamily]{Ligatures=TeX,Scale=1}
\fi
% Use upquote if available, for straight quotes in verbatim environments
\IfFileExists{upquote.sty}{\usepackage{upquote}}{}
\IfFileExists{microtype.sty}{% use microtype if available
  \usepackage[]{microtype}
  \UseMicrotypeSet[protrusion]{basicmath} % disable protrusion for tt fonts
}{}
\makeatletter
\@ifundefined{KOMAClassName}{% if non-KOMA class
  \IfFileExists{parskip.sty}{%
    \usepackage{parskip}
  }{% else
    \setlength{\parindent}{0pt}
    \setlength{\parskip}{6pt plus 2pt minus 1pt}}
}{% if KOMA class
  \KOMAoptions{parskip=half}}
\makeatother
\usepackage{xcolor}
\IfFileExists{xurl.sty}{\usepackage{xurl}}{} % add URL line breaks if available
\IfFileExists{bookmark.sty}{\usepackage{bookmark}}{\usepackage{hyperref}}
\hypersetup{
  pdftitle={NHA\_LAB36},
  pdfauthor={Nancy Hernández Ambris},
  hidelinks,
  pdfcreator={LaTeX via pandoc}}
\urlstyle{same} % disable monospaced font for URLs
\usepackage[margin=1in]{geometry}
\usepackage{color}
\usepackage{fancyvrb}
\newcommand{\VerbBar}{|}
\newcommand{\VERB}{\Verb[commandchars=\\\{\}]}
\DefineVerbatimEnvironment{Highlighting}{Verbatim}{commandchars=\\\{\}}
% Add ',fontsize=\small' for more characters per line
\usepackage{framed}
\definecolor{shadecolor}{RGB}{248,248,248}
\newenvironment{Shaded}{\begin{snugshade}}{\end{snugshade}}
\newcommand{\AlertTok}[1]{\textcolor[rgb]{0.94,0.16,0.16}{#1}}
\newcommand{\AnnotationTok}[1]{\textcolor[rgb]{0.56,0.35,0.01}{\textbf{\textit{#1}}}}
\newcommand{\AttributeTok}[1]{\textcolor[rgb]{0.77,0.63,0.00}{#1}}
\newcommand{\BaseNTok}[1]{\textcolor[rgb]{0.00,0.00,0.81}{#1}}
\newcommand{\BuiltInTok}[1]{#1}
\newcommand{\CharTok}[1]{\textcolor[rgb]{0.31,0.60,0.02}{#1}}
\newcommand{\CommentTok}[1]{\textcolor[rgb]{0.56,0.35,0.01}{\textit{#1}}}
\newcommand{\CommentVarTok}[1]{\textcolor[rgb]{0.56,0.35,0.01}{\textbf{\textit{#1}}}}
\newcommand{\ConstantTok}[1]{\textcolor[rgb]{0.00,0.00,0.00}{#1}}
\newcommand{\ControlFlowTok}[1]{\textcolor[rgb]{0.13,0.29,0.53}{\textbf{#1}}}
\newcommand{\DataTypeTok}[1]{\textcolor[rgb]{0.13,0.29,0.53}{#1}}
\newcommand{\DecValTok}[1]{\textcolor[rgb]{0.00,0.00,0.81}{#1}}
\newcommand{\DocumentationTok}[1]{\textcolor[rgb]{0.56,0.35,0.01}{\textbf{\textit{#1}}}}
\newcommand{\ErrorTok}[1]{\textcolor[rgb]{0.64,0.00,0.00}{\textbf{#1}}}
\newcommand{\ExtensionTok}[1]{#1}
\newcommand{\FloatTok}[1]{\textcolor[rgb]{0.00,0.00,0.81}{#1}}
\newcommand{\FunctionTok}[1]{\textcolor[rgb]{0.00,0.00,0.00}{#1}}
\newcommand{\ImportTok}[1]{#1}
\newcommand{\InformationTok}[1]{\textcolor[rgb]{0.56,0.35,0.01}{\textbf{\textit{#1}}}}
\newcommand{\KeywordTok}[1]{\textcolor[rgb]{0.13,0.29,0.53}{\textbf{#1}}}
\newcommand{\NormalTok}[1]{#1}
\newcommand{\OperatorTok}[1]{\textcolor[rgb]{0.81,0.36,0.00}{\textbf{#1}}}
\newcommand{\OtherTok}[1]{\textcolor[rgb]{0.56,0.35,0.01}{#1}}
\newcommand{\PreprocessorTok}[1]{\textcolor[rgb]{0.56,0.35,0.01}{\textit{#1}}}
\newcommand{\RegionMarkerTok}[1]{#1}
\newcommand{\SpecialCharTok}[1]{\textcolor[rgb]{0.00,0.00,0.00}{#1}}
\newcommand{\SpecialStringTok}[1]{\textcolor[rgb]{0.31,0.60,0.02}{#1}}
\newcommand{\StringTok}[1]{\textcolor[rgb]{0.31,0.60,0.02}{#1}}
\newcommand{\VariableTok}[1]{\textcolor[rgb]{0.00,0.00,0.00}{#1}}
\newcommand{\VerbatimStringTok}[1]{\textcolor[rgb]{0.31,0.60,0.02}{#1}}
\newcommand{\WarningTok}[1]{\textcolor[rgb]{0.56,0.35,0.01}{\textbf{\textit{#1}}}}
\usepackage{graphicx}
\makeatletter
\def\maxwidth{\ifdim\Gin@nat@width>\linewidth\linewidth\else\Gin@nat@width\fi}
\def\maxheight{\ifdim\Gin@nat@height>\textheight\textheight\else\Gin@nat@height\fi}
\makeatother
% Scale images if necessary, so that they will not overflow the page
% margins by default, and it is still possible to overwrite the defaults
% using explicit options in \includegraphics[width, height, ...]{}
\setkeys{Gin}{width=\maxwidth,height=\maxheight,keepaspectratio}
% Set default figure placement to htbp
\makeatletter
\def\fps@figure{htbp}
\makeatother
\setlength{\emergencystretch}{3em} % prevent overfull lines
\providecommand{\tightlist}{%
  \setlength{\itemsep}{0pt}\setlength{\parskip}{0pt}}
\setcounter{secnumdepth}{-\maxdimen} % remove section numbering
\ifLuaTeX
  \usepackage{selnolig}  % disable illegal ligatures
\fi

\begin{document}
\maketitle

Laboratorio - MAPA DE CALOR -TÉRMICO- with pheatmap DATOS GENETICOS
TOMADOS DE Sahir Bhatnagar.PRACTICA DE CODERS

Objetivo: Realizar un heatmap con datos geneticos
-------------------------------------------------------------------------------------
En este ejercicio vamos a: 1. Cargar nuestra matriz hipotética de datos
y dataframes adicionales 2. Realizar varios heatmaps

Un mapa de calor es una representación gráfica de datos que utiliza un
sistema de codificación de colores para representar diferentes valores

Heatmaps with pheatmap Simulated data created by Sahir Bhatnagar.

\#install.packages(``pheatmap'')

\begin{Shaded}
\begin{Highlighting}[]
\FunctionTok{library}\NormalTok{(pheatmap)}
\end{Highlighting}
\end{Shaded}

\hypertarget{importar-datos}{%
\section{importar datos}\label{importar-datos}}

\begin{Shaded}
\begin{Highlighting}[]
\FunctionTok{file.choose}\NormalTok{()}
\end{Highlighting}
\end{Shaded}

\begin{verbatim}
## [1] "C:\\Users\\Nancy\\Desktop\\Doctorado\\Asignaturas\\Complejidad económica\\Labs\\LAB36\\NHA_LAB36_.html"
\end{verbatim}

\begin{Shaded}
\begin{Highlighting}[]
\NormalTok{genes }\OtherTok{\textless{}{-}} \FunctionTok{as.matrix}\NormalTok{(}\FunctionTok{read.csv}\NormalTok{(}\StringTok{"C:}\SpecialCharTok{\textbackslash{}\textbackslash{}}\StringTok{Users}\SpecialCharTok{\textbackslash{}\textbackslash{}}\StringTok{Nancy}\SpecialCharTok{\textbackslash{}\textbackslash{}}\StringTok{Desktop}\SpecialCharTok{\textbackslash{}\textbackslash{}}\StringTok{Doctorado}\SpecialCharTok{\textbackslash{}\textbackslash{}}\StringTok{Asignaturas}\SpecialCharTok{\textbackslash{}\textbackslash{}}\StringTok{Complejidad económica}\SpecialCharTok{\textbackslash{}\textbackslash{}}\StringTok{Labs}\SpecialCharTok{\textbackslash{}\textbackslash{}}\StringTok{LAB36}\SpecialCharTok{\textbackslash{}\textbackslash{}}\StringTok{heatmap\_data.csv"}\NormalTok{,}
                            \AttributeTok{sep =} \StringTok{","}\NormalTok{,}
                            \AttributeTok{header =}\NormalTok{ T,}
                            \AttributeTok{row.names =} \DecValTok{1}\NormalTok{))}
\end{Highlighting}
\end{Shaded}

\begin{Shaded}
\begin{Highlighting}[]
\NormalTok{annotation\_col }\OtherTok{\textless{}{-}} \FunctionTok{read.csv}\NormalTok{(}\StringTok{"C:}\SpecialCharTok{\textbackslash{}\textbackslash{}}\StringTok{Users}\SpecialCharTok{\textbackslash{}\textbackslash{}}\StringTok{Nancy}\SpecialCharTok{\textbackslash{}\textbackslash{}}\StringTok{Desktop}\SpecialCharTok{\textbackslash{}\textbackslash{}}\StringTok{Doctorado}\SpecialCharTok{\textbackslash{}\textbackslash{}}\StringTok{Asignaturas}\SpecialCharTok{\textbackslash{}\textbackslash{}}\StringTok{Complejidad económica}\SpecialCharTok{\textbackslash{}\textbackslash{}}\StringTok{Labs}\SpecialCharTok{\textbackslash{}\textbackslash{}}\StringTok{LAB36}\SpecialCharTok{\textbackslash{}\textbackslash{}}\StringTok{annotation\_col.csv"}\NormalTok{,}
                           \AttributeTok{row.names =} \DecValTok{1}\NormalTok{)}
\end{Highlighting}
\end{Shaded}

\begin{Shaded}
\begin{Highlighting}[]
\NormalTok{annotation\_row }\OtherTok{\textless{}{-}} \FunctionTok{read.csv}\NormalTok{(}\StringTok{"C:}\SpecialCharTok{\textbackslash{}\textbackslash{}}\StringTok{Users}\SpecialCharTok{\textbackslash{}\textbackslash{}}\StringTok{Nancy}\SpecialCharTok{\textbackslash{}\textbackslash{}}\StringTok{Desktop}\SpecialCharTok{\textbackslash{}\textbackslash{}}\StringTok{Doctorado}\SpecialCharTok{\textbackslash{}\textbackslash{}}\StringTok{Asignaturas}\SpecialCharTok{\textbackslash{}\textbackslash{}}\StringTok{Complejidad económica}\SpecialCharTok{\textbackslash{}\textbackslash{}}\StringTok{Labs}\SpecialCharTok{\textbackslash{}\textbackslash{}}\StringTok{LAB36}\SpecialCharTok{\textbackslash{}\textbackslash{}}\StringTok{annotation\_row.csv"}\NormalTok{,}
                           \AttributeTok{row.names =} \DecValTok{1}\NormalTok{)}
\end{Highlighting}
\end{Shaded}

Plotting with pheatmap!

\begin{Shaded}
\begin{Highlighting}[]
\FunctionTok{pheatmap}\NormalTok{(genes)}
\end{Highlighting}
\end{Shaded}

\includegraphics{NHA_LAB36__files/figure-latex/unnamed-chunk-5-1.pdf}
change font

\begin{Shaded}
\begin{Highlighting}[]
\FunctionTok{pheatmap}\NormalTok{(genes, }\AttributeTok{frontsize =} \DecValTok{6}\NormalTok{)}
\end{Highlighting}
\end{Shaded}

\includegraphics{NHA_LAB36__files/figure-latex/unnamed-chunk-6-1.pdf}
default is clustering rows and columns cluster by gene - groups of
similar genes----LOS GENES ESTAN EN LOS RENGLONES

POR DEFAULT CLUSTEA LOS RENGLONES

\begin{Shaded}
\begin{Highlighting}[]
\FunctionTok{pheatmap}\NormalTok{(genes, }\AttributeTok{frontsize =} \DecValTok{6}\NormalTok{, }\AttributeTok{cluster\_rows =}\NormalTok{ T, }\AttributeTok{cluster\_cols =}\NormalTok{ F)}
\end{Highlighting}
\end{Shaded}

\includegraphics{NHA_LAB36__files/figure-latex/unnamed-chunk-7-1.pdf}

cluster by patient - groups of similar patients DEBES HACER QUE LAS
COLUMNAS SE TRANFOMEN A RENGLONES

\begin{Shaded}
\begin{Highlighting}[]
\FunctionTok{pheatmap}\NormalTok{(genes, }\AttributeTok{frontsize =} \DecValTok{6}\NormalTok{, }\AttributeTok{cluster\_rows =}\NormalTok{ F, }\AttributeTok{cluster\_cols =}\NormalTok{ T)}
\end{Highlighting}
\end{Shaded}

\includegraphics{NHA_LAB36__files/figure-latex/unnamed-chunk-8-1.pdf}
usually order by both

\begin{Shaded}
\begin{Highlighting}[]
\FunctionTok{pheatmap}\NormalTok{(genes, }\AttributeTok{frontsize =} \DecValTok{6}\NormalTok{, }\AttributeTok{cluster\_rows =}\NormalTok{ T, }\AttributeTok{cluster\_cols =}\NormalTok{ T)}
\end{Highlighting}
\end{Shaded}

\includegraphics{NHA_LAB36__files/figure-latex/unnamed-chunk-9-1.pdf}
seeing some patterns emerge - but what do they mean? Great time to add
annotation to our plot

\begin{Shaded}
\begin{Highlighting}[]
\FunctionTok{pheatmap}\NormalTok{(genes, }\AttributeTok{frontsize =} \DecValTok{6}\NormalTok{, }\AttributeTok{cluster\_rows =}\NormalTok{ T, }\AttributeTok{cluster\_cols =}\NormalTok{ T,}\AttributeTok{annotation\_row =}\NormalTok{ annotation\_row)}
\end{Highlighting}
\end{Shaded}

\includegraphics{NHA_LAB36__files/figure-latex/unnamed-chunk-10-1.pdf}
add to row first, see that genes are clustering according to the
pathways they belong to

\begin{Shaded}
\begin{Highlighting}[]
\FunctionTok{pheatmap}\NormalTok{(genes, }\AttributeTok{frontsize =} \DecValTok{6}\NormalTok{, }\AttributeTok{cluster\_rows =}\NormalTok{ T, }\AttributeTok{cluster\_cols =}\NormalTok{ T,}\AttributeTok{annotation\_row =}\NormalTok{ annotation\_row, }\AttributeTok{annotation\_col =}\NormalTok{ annotation\_col)}
\end{Highlighting}
\end{Shaded}

\includegraphics{NHA_LAB36__files/figure-latex/unnamed-chunk-11-1.pdf}
now have information about the drug and condition

GRAFICO COMPLETO G1

\begin{Shaded}
\begin{Highlighting}[]
\FunctionTok{pheatmap}\NormalTok{(genes, }\AttributeTok{frontsize =} \DecValTok{6}\NormalTok{, }\AttributeTok{cluster\_rows =}\NormalTok{ T, }\AttributeTok{cluster\_cols =}\NormalTok{ T,}\AttributeTok{annotation\_row =}\NormalTok{ annotation\_row, }\AttributeTok{annotation\_col =}\NormalTok{ annotation\_col, }\AttributeTok{treeheight\_row =} \DecValTok{0}\NormalTok{,}\AttributeTok{treeheight\_col =} \DecValTok{0}\NormalTok{, }\AttributeTok{main =} \StringTok{"expresion genetica"}\NormalTok{)}
\end{Highlighting}
\end{Shaded}

\includegraphics{NHA_LAB36__files/figure-latex/unnamed-chunk-12-1.pdf}
GRAFICO QUITANDO CLUSTERS (ARBOLES DE AGRUPACIÓN-DENDOGRAMAS) take a
smaller subset

\begin{Shaded}
\begin{Highlighting}[]
\NormalTok{sub }\OtherTok{\textless{}{-}}\NormalTok{ genes[}\FunctionTok{c}\NormalTok{(}\DecValTok{1}\SpecialCharTok{:}\DecValTok{5}\NormalTok{, }\DecValTok{55}\SpecialCharTok{:}\DecValTok{60}\NormalTok{), }\FunctionTok{c}\NormalTok{(}\DecValTok{1}\SpecialCharTok{:}\DecValTok{5}\NormalTok{, }\DecValTok{20}\SpecialCharTok{:}\DecValTok{35}\NormalTok{, }\DecValTok{55}\SpecialCharTok{:}\DecValTok{60}\NormalTok{)]}
\end{Highlighting}
\end{Shaded}

con subset 1 (COPIAR G1)

\begin{Shaded}
\begin{Highlighting}[]
\FunctionTok{pheatmap}\NormalTok{(sub, }\AttributeTok{frontsize =} \DecValTok{6}\NormalTok{, }\AttributeTok{cluster\_rows =}\NormalTok{ T, }\AttributeTok{cluster\_cols =}\NormalTok{ T,}\AttributeTok{annotation\_row =}\NormalTok{ annotation\_row, }\AttributeTok{annotation\_col =}\NormalTok{ annotation\_col, }\AttributeTok{treeheight\_row =} \DecValTok{0}\NormalTok{,}\AttributeTok{treeheight\_col =} \DecValTok{0}\NormalTok{, }\AttributeTok{main =} \StringTok{"expresion genetica"}\NormalTok{)}
\end{Highlighting}
\end{Shaded}

\includegraphics{NHA_LAB36__files/figure-latex/unnamed-chunk-14-1.pdf}
con subset 2 -- DESPLEGAR VALORES

\begin{Shaded}
\begin{Highlighting}[]
\FunctionTok{pheatmap}\NormalTok{(sub, }\AttributeTok{frontsize =} \DecValTok{6}\NormalTok{, }\AttributeTok{cluster\_rows =}\NormalTok{ T, }\AttributeTok{cluster\_cols =}\NormalTok{ T,}\AttributeTok{annotation\_row =}\NormalTok{ annotation\_row, }\AttributeTok{annotation\_col =}\NormalTok{ annotation\_col, }\AttributeTok{treeheight\_row =} \DecValTok{0}\NormalTok{,}\AttributeTok{treeheight\_col =} \DecValTok{0}\NormalTok{, }\AttributeTok{main =} \StringTok{"expresion genetica"}\NormalTok{, }\AttributeTok{fontsize =} \DecValTok{8}\NormalTok{, }\AttributeTok{annotation\_legend =} \ConstantTok{FALSE}\NormalTok{, }\AttributeTok{display\_numbers =} \ConstantTok{TRUE}\NormalTok{, }\AttributeTok{fontsize\_number =} \DecValTok{6}\NormalTok{)}
\end{Highlighting}
\end{Shaded}

\includegraphics{NHA_LAB36__files/figure-latex/unnamed-chunk-15-1.pdf}
viridis, magma, plasma, cividis, inferno

\begin{Shaded}
\begin{Highlighting}[]
\FunctionTok{library}\NormalTok{(viridis)}
\end{Highlighting}
\end{Shaded}

\begin{verbatim}
## Loading required package: viridisLite
\end{verbatim}

con color

\begin{Shaded}
\begin{Highlighting}[]
\FunctionTok{pheatmap}\NormalTok{(sub, }\AttributeTok{frontsize =} \DecValTok{6}\NormalTok{, }\AttributeTok{cluster\_rows =}\NormalTok{ T, }\AttributeTok{cluster\_cols =}\NormalTok{ T,}\AttributeTok{annotation\_row =}\NormalTok{ annotation\_row, }\AttributeTok{annotation\_col =}\NormalTok{ annotation\_col, }\AttributeTok{treeheight\_row =} \DecValTok{0}\NormalTok{,}\AttributeTok{treeheight\_col =} \DecValTok{0}\NormalTok{, }\AttributeTok{main =} \StringTok{"expresion genetica"}\NormalTok{, }\AttributeTok{fontsize =} \DecValTok{8}\NormalTok{, }\AttributeTok{annotation\_legend =} \ConstantTok{FALSE}\NormalTok{, }\AttributeTok{display\_numbers =} \ConstantTok{TRUE}\NormalTok{, }\AttributeTok{fontsize\_number =} \DecValTok{6}\NormalTok{,}\AttributeTok{col =} \FunctionTok{viridis\_pal}\NormalTok{(}\AttributeTok{option =} \StringTok{"viridis"}\NormalTok{)(}\DecValTok{6}\NormalTok{))}
\end{Highlighting}
\end{Shaded}

\includegraphics{NHA_LAB36__files/figure-latex/unnamed-chunk-17-1.pdf}
elementos adicionales

\begin{Shaded}
\begin{Highlighting}[]
\FunctionTok{dist}\NormalTok{(sub)}
\end{Highlighting}
\end{Shaded}

\begin{verbatim}
##           Gene1    Gene2    Gene3    Gene4    Gene5   Gene55   Gene56   Gene57
## Gene2  6.506125                                                               
## Gene3  7.823569 7.021725                                                      
## Gene4  5.253565 7.649124 6.516104                                             
## Gene5  6.411847 5.977640 5.967513 6.184570                                    
## Gene55 5.703940 6.969997 7.096321 6.837653 7.534618                           
## Gene56 4.544832 6.723925 6.542745 5.805165 5.150859 6.028094                  
## Gene57 6.124657 6.069362 5.550487 6.004035 3.881691 7.122986 5.209746         
## Gene58 7.417422 8.796956 8.462521 7.874145 8.030439 6.777444 6.292359 7.669524
## Gene59 6.189649 8.293720 7.977707 6.115718 5.821355 7.317126 4.835770 6.104449
## Gene60 6.623226 8.133474 7.665999 6.837342 7.659167 7.569942 6.373711 7.296198
##          Gene58   Gene59
## Gene2                   
## Gene3                   
## Gene4                   
## Gene5                   
## Gene55                  
## Gene56                  
## Gene57                  
## Gene58                  
## Gene59 8.312043         
## Gene60 7.813793 6.992657
\end{verbatim}

\begin{Shaded}
\begin{Highlighting}[]
\FunctionTok{pheatmap}\NormalTok{(}\FunctionTok{cor}\NormalTok{(sub))}
\end{Highlighting}
\end{Shaded}

\includegraphics{NHA_LAB36__files/figure-latex/unnamed-chunk-18-1.pdf}

\begin{Shaded}
\begin{Highlighting}[]
\NormalTok{trans }\OtherTok{\textless{}{-}} \FunctionTok{t}\NormalTok{(sub)}
\FunctionTok{pheatmap}\NormalTok{(}\FunctionTok{cor}\NormalTok{(trans))}
\end{Highlighting}
\end{Shaded}

\includegraphics{NHA_LAB36__files/figure-latex/unnamed-chunk-18-2.pdf}
Crear cuaderno de R Markdown

\end{document}
